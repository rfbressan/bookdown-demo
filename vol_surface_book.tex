\documentclass[]{book}
\usepackage{lmodern}
\usepackage{amssymb,amsmath}
\usepackage{ifxetex,ifluatex}
\usepackage{fixltx2e} % provides \textsubscript
\ifnum 0\ifxetex 1\fi\ifluatex 1\fi=0 % if pdftex
  \usepackage[T1]{fontenc}
  \usepackage[utf8]{inputenc}
\else % if luatex or xelatex
  \ifxetex
    \usepackage{mathspec}
  \else
    \usepackage{fontspec}
  \fi
  \defaultfontfeatures{Ligatures=TeX,Scale=MatchLowercase}
\fi
% use upquote if available, for straight quotes in verbatim environments
\IfFileExists{upquote.sty}{\usepackage{upquote}}{}
% use microtype if available
\IfFileExists{microtype.sty}{%
\usepackage{microtype}
\UseMicrotypeSet[protrusion]{basicmath} % disable protrusion for tt fonts
}{}
\usepackage[margin=1in]{geometry}
\usepackage{hyperref}
\hypersetup{unicode=true,
            pdftitle={Introdução a Superfície de Volatilidade},
            pdfauthor={Rafael Felipe Bressan},
            pdfborder={0 0 0},
            breaklinks=true}
\urlstyle{same}  % don't use monospace font for urls
\usepackage{natbib}
\bibliographystyle{apalike}
\usepackage{color}
\usepackage{fancyvrb}
\newcommand{\VerbBar}{|}
\newcommand{\VERB}{\Verb[commandchars=\\\{\}]}
\DefineVerbatimEnvironment{Highlighting}{Verbatim}{commandchars=\\\{\}}
% Add ',fontsize=\small' for more characters per line
\usepackage{framed}
\definecolor{shadecolor}{RGB}{248,248,248}
\newenvironment{Shaded}{\begin{snugshade}}{\end{snugshade}}
\newcommand{\KeywordTok}[1]{\textcolor[rgb]{0.13,0.29,0.53}{\textbf{#1}}}
\newcommand{\DataTypeTok}[1]{\textcolor[rgb]{0.13,0.29,0.53}{#1}}
\newcommand{\DecValTok}[1]{\textcolor[rgb]{0.00,0.00,0.81}{#1}}
\newcommand{\BaseNTok}[1]{\textcolor[rgb]{0.00,0.00,0.81}{#1}}
\newcommand{\FloatTok}[1]{\textcolor[rgb]{0.00,0.00,0.81}{#1}}
\newcommand{\ConstantTok}[1]{\textcolor[rgb]{0.00,0.00,0.00}{#1}}
\newcommand{\CharTok}[1]{\textcolor[rgb]{0.31,0.60,0.02}{#1}}
\newcommand{\SpecialCharTok}[1]{\textcolor[rgb]{0.00,0.00,0.00}{#1}}
\newcommand{\StringTok}[1]{\textcolor[rgb]{0.31,0.60,0.02}{#1}}
\newcommand{\VerbatimStringTok}[1]{\textcolor[rgb]{0.31,0.60,0.02}{#1}}
\newcommand{\SpecialStringTok}[1]{\textcolor[rgb]{0.31,0.60,0.02}{#1}}
\newcommand{\ImportTok}[1]{#1}
\newcommand{\CommentTok}[1]{\textcolor[rgb]{0.56,0.35,0.01}{\textit{#1}}}
\newcommand{\DocumentationTok}[1]{\textcolor[rgb]{0.56,0.35,0.01}{\textbf{\textit{#1}}}}
\newcommand{\AnnotationTok}[1]{\textcolor[rgb]{0.56,0.35,0.01}{\textbf{\textit{#1}}}}
\newcommand{\CommentVarTok}[1]{\textcolor[rgb]{0.56,0.35,0.01}{\textbf{\textit{#1}}}}
\newcommand{\OtherTok}[1]{\textcolor[rgb]{0.56,0.35,0.01}{#1}}
\newcommand{\FunctionTok}[1]{\textcolor[rgb]{0.00,0.00,0.00}{#1}}
\newcommand{\VariableTok}[1]{\textcolor[rgb]{0.00,0.00,0.00}{#1}}
\newcommand{\ControlFlowTok}[1]{\textcolor[rgb]{0.13,0.29,0.53}{\textbf{#1}}}
\newcommand{\OperatorTok}[1]{\textcolor[rgb]{0.81,0.36,0.00}{\textbf{#1}}}
\newcommand{\BuiltInTok}[1]{#1}
\newcommand{\ExtensionTok}[1]{#1}
\newcommand{\PreprocessorTok}[1]{\textcolor[rgb]{0.56,0.35,0.01}{\textit{#1}}}
\newcommand{\AttributeTok}[1]{\textcolor[rgb]{0.77,0.63,0.00}{#1}}
\newcommand{\RegionMarkerTok}[1]{#1}
\newcommand{\InformationTok}[1]{\textcolor[rgb]{0.56,0.35,0.01}{\textbf{\textit{#1}}}}
\newcommand{\WarningTok}[1]{\textcolor[rgb]{0.56,0.35,0.01}{\textbf{\textit{#1}}}}
\newcommand{\AlertTok}[1]{\textcolor[rgb]{0.94,0.16,0.16}{#1}}
\newcommand{\ErrorTok}[1]{\textcolor[rgb]{0.64,0.00,0.00}{\textbf{#1}}}
\newcommand{\NormalTok}[1]{#1}
\usepackage{longtable,booktabs}
\usepackage{graphicx,grffile}
\makeatletter
\def\maxwidth{\ifdim\Gin@nat@width>\linewidth\linewidth\else\Gin@nat@width\fi}
\def\maxheight{\ifdim\Gin@nat@height>\textheight\textheight\else\Gin@nat@height\fi}
\makeatother
% Scale images if necessary, so that they will not overflow the page
% margins by default, and it is still possible to overwrite the defaults
% using explicit options in \includegraphics[width, height, ...]{}
\setkeys{Gin}{width=\maxwidth,height=\maxheight,keepaspectratio}
\IfFileExists{parskip.sty}{%
\usepackage{parskip}
}{% else
\setlength{\parindent}{0pt}
\setlength{\parskip}{6pt plus 2pt minus 1pt}
}
\setlength{\emergencystretch}{3em}  % prevent overfull lines
\providecommand{\tightlist}{%
  \setlength{\itemsep}{0pt}\setlength{\parskip}{0pt}}
\setcounter{secnumdepth}{5}
% Redefines (sub)paragraphs to behave more like sections
\ifx\paragraph\undefined\else
\let\oldparagraph\paragraph
\renewcommand{\paragraph}[1]{\oldparagraph{#1}\mbox{}}
\fi
\ifx\subparagraph\undefined\else
\let\oldsubparagraph\subparagraph
\renewcommand{\subparagraph}[1]{\oldsubparagraph{#1}\mbox{}}
\fi

%%% Use protect on footnotes to avoid problems with footnotes in titles
\let\rmarkdownfootnote\footnote%
\def\footnote{\protect\rmarkdownfootnote}

%%% Change title format to be more compact
\usepackage{titling}

% Create subtitle command for use in maketitle
\newcommand{\subtitle}[1]{
  \posttitle{
    \begin{center}\large#1\end{center}
    }
}

\setlength{\droptitle}{-2em}

  \title{Introdução a Superfície de Volatilidade}
    \pretitle{\vspace{\droptitle}\centering\huge}
  \posttitle{\par}
    \author{Rafael Felipe Bressan}
    \preauthor{\centering\large\emph}
  \postauthor{\par}
      \predate{\centering\large\emph}
  \postdate{\par}
    \date{2019-01-10}

\usepackage{booktabs}
\usepackage{amsthm}
\usepackage[english, portuges]{babel}
\usepackage{indentfirst}
% O tamanho da identação do parágrafo é dado por:
\setlength{\parindent}{2.5cm}
\makeatletter
\def\thm@space@setup{%
  \thm@preskip=8pt plus 2pt minus 4pt
  \thm@postskip=\thm@preskip
}
\makeatother
\usepackage{booktabs}
\usepackage{longtable}
\usepackage{array}
\usepackage{multirow}
\usepackage[table]{xcolor}
\usepackage{wrapfig}
\usepackage{float}
\usepackage{colortbl}
\usepackage{pdflscape}
\usepackage{tabu}
\usepackage{threeparttable}
\usepackage{threeparttablex}
\usepackage[normalem]{ulem}
\usepackage{makecell}

\begin{document}
\maketitle

{
\setcounter{tocdepth}{1}
\tableofcontents
}
\chapter*{Prefácio}\label{prefacio}
\addcontentsline{toc}{chapter}{Prefácio}

Este é um pequeno resumo, elaborado na forma de livro, sobre os estudos
realizados pelo núcleo de derivativos e riscos do
\href{https://clubedefinancas.com.br}{Clube de finanças ESAG}.

Os estudos realizados pelos membros do núcleo foram sendo apresentados
na forma de artigos no blog do Clube. A partir destes artigos foi feita
esta coletânea de forma a apresentar todo o conteúdo em local único para
facilitar a assimilação dos membros futuros e leitores de nosso blog.

\chapter*{Sobre os Autores}\label{sobre-os-autores}
\addcontentsline{toc}{chapter}{Sobre os Autores}

\section*{Rafael Felipe Bressan}\label{rafael-felipe-bressan}
\addcontentsline{toc}{section}{Rafael Felipe Bressan}

Formado em Engenharia de Controle e Automação Industrial pela UFSC e
aluno de graduação do curso de Ciências Econômicas na UDESC/Esag. Membro
do Clube de Finanças Esag e gerente do núcleo de pesquisa em riscos e
derivativos.

Se interessa por finanças quantitativas, modelagem e controle de riscos
e desenvolveu, durante a elaboração deste livro, grande curiosidade
sobre precificação de derivativos. Gosta de programar em
\href{https://cran.r-project.org/}{R}, liguagem com a qual elaborou este
próprio livro e está aprendendo \href{https://www.python.org/}{Python}.

\chapter{Introdução as opções}\label{opcoes}

Neste Capítulo\footnote{Esta é uma suposição do modelo, pode não ser
  verdade para algum investidor qualquer mas se for para algum outro
  investidor representativo, o princípio de ausência de arbitragem passa
  a valer, uma vez que este segundo investidor explorará o mercado e
  levará o preço do derivativo para o resultado requerido.} iremos
apresentar os instrumentos financeiros conhecidos como opções. Existem
dois tipos de opção. Uma opção de compra dá ao detentor o direito de
comprar o ativo subjacente até uma determinada data por um determinado
preço. Uma opção de venda dá ao titular o direito de vender o ativo
subjacente até uma determinada data por um determinado preço.

Alguns termos importantes para o melhor entendimento deste Capítulo são:
``Ativo subjacente'', que é o ativo negociado no contrato, ``data de
vencimento'', no caso do modelo americano é até a data limite para
exercer a opção de compra e no modelo europeu é na data final em que a
opção de compra pode ou não ser exercida (serão discutidos mais detalhes
sobre estas modalidades posteriormente) e ``preço de exercício''
(strike), é o valor a ser pago pelo ativo de acordo com o contrato. As
opções americanas podem ser exercidas a qualquer momento até a data de
vencimento. Por sua vez, as opções europeias podem ser exercidas somente
na própria data de vencimento.

As opções europeias são geralmente mais fáceis de analisar do que as
opções americanas, e algumas das propriedades de uma opção americana são
frequentemente deduzidas daquelas de sua contraparte europeia.

A opção de compra (call) dá ao comprador da opção o \textbf{direito de
comprar} o ativo subjacente pelo preço de exercício. A opção de venda
(put) dá ao comprador da opção o \textbf{direito de vender} o ativo
subjacente pelo preço combinado no contrato na data futura.

O preço de uma opção de compra diminui à medida que o preço de exercício
aumenta, enquanto o preço de uma opção de venda aumenta à medida que o
preço de exercício aumenta. Ambos os tipos de opção tendem a se tornar
mais valiosas à medida que seu tempo até o vencimento aumenta. Na
verdade existem seis fatores que afetam o preço de uma opção de ação,
por exemplo:

\begin{enumerate}
\def\labelenumi{\arabic{enumi}.}
\tightlist
\item
  O preço atual da ação, \(S_0\)
\item
  O preço de exercício, \(K\)
\item
  O tempo para expiração, \(\tau\)
\item
  A volatilidade do preço das ações, \(\sigma\)
\item
  A taxa de juros livre de risco, \(r\)
\item
  Os dividendos que se espera sejam pagos, \(q\)
\end{enumerate}

Observamos que existem quatro tipos de participantes nos mercados de
opções:

\begin{itemize}
\tightlist
\item
  Compradores de calls
\item
  Vendedores de calls
\item
  Compradores de puts
\item
  Vendedores de puts
\end{itemize}

Os compradores são referidos como tendo posições \emph{long}, vendedores
são referidos como tendo posições \emph{short}. Vender uma opção também
é conhecido como lançar a opção ou subscrever.

Exemplificando uma operação de compra de call, caso o preço do ativo
tenha subido acima do preço de strike o comprador pode exercer sua opção
de compra e ele lucrará a partir do momento em que o valor da ação for
maior que o strike mais o valor pago pela opção (chamado de prêmio).
Caso o preço do ativo tenha caído abaixo do strike, o comprador
simplesmente não exerce sua opção, limitando sua perda nessa operação ao
prêmio pago.

\textbackslash{}begin\{table\}{[}t{]}

\textbackslash{}caption\{\label{tab:opcao}Exemplo: Comprando call de X com o
preço de exercício (strike) de R\$ 10.000\} \centering

\begin{tabular}{r|r|r|r|l}
\hline
Ação & Prêmio Pago & Opção no Exercício & Lucro & Moneyness\\
\hline
9000 & 200 & 0 & -200 & OTM\\
\hline
9500 & 200 & 0 & -200 & OTM\\
\hline
10000 & 200 & 0 & -200 & ATM\\
\hline
10200 & 200 & 200 & 0 & ITM\\
\hline
10500 & 200 & 500 & 300 & ITM\\
\hline
11000 & 200 & 1000 & 800 & ITM\\
\hline
\end{tabular}

\textbackslash{}end\{table\}

Usando a tabela \ref{tab:opcao} como exemplo é possível ver que o
resultado final será maior que R\$ 0 quando o valor do ativo subjacente
é maior que R\$10.200 (R\$10.000 de strike + R\$200 de prêmio), e o
prejuízo final está limitado ao valor de R\$200. A figura \ref{fig:call}
abaixo mostra o perfil de lucro da operação exemplificada, típico de uma
compra de call.

\begin{figure}
\centering
\includegraphics{vol_surface_book_files/figure-latex/call-1.pdf}
\caption{\label{fig:call}Perfil de lucro típico de uma compra de call.}
\end{figure}

No caso de uma operação de uma compra de put, caso o preço do ativo
tenha subido acima do strike, não faz sentindo o detentor da opção
exercer seu direito, assim sua perda será apenas o valor pago pela
opção. Caso o preço do ativo tenha descido abaixo do strike, o comprador
da put pode realizar a venda e começará a lucrar a partir do momento em
que o strike fique acima do valor do ativo somado ao valor do prêmio
pago pela opção. Usando a tabela 1.2 como exemplo é possível ver que o
resultado final será de no mínimo -R\$600 caso o valor do ativo
subjacente seja igual ou maior que R\$15.000, e o resultado final
aumenta conforme o ativo perde o valor, sendo positivo a partir de
quando seu valor é de R\$14.400.

\section{Conceitos in the money, at the money e out the
money}\label{conceitos-in-the-money-at-the-money-e-out-the-money}

Estes termos são usados para se referir a opções que estão com o preço
de exercício (strike) do ativo abaixo, acima ou igual ao valor atual do
ativo.

\begin{itemize}
\tightlist
\item
  Out the money: Strike do ativo subjacente está abaixo do valor de
  mercado no caso de calls ou quando o strike do ativo está acima do
  valor de mercado no caso de puts.\\
\item
  At the money: Strike do ativo subjacente é o igual ao valor de
  mercado.
\item
  In the money: Strike do ativo subjacente está acima do valor de
  mercado no caso de calls ou quando o strike do ativo está abaixo do
  valor de mercado no caso de puts.
\end{itemize}

\section{Modelos americano e europeu de
opções}\label{modelos-americano-e-europeu-de-opcoes}

No modelo americano de opções o comprador pode exercer seu direito de
compra ou venda do ativo subjacente a qualquer momento entre o início do
contrato e o vencimento dele, enquanto isso no modelo europeu a
transação só pode ser realizada na data de vencimento.

\section{Hedge}\label{hedge}

O mercado de opções pode ser usado tanto para hedge (proteção) quanto
para especulação. O hedge é feito para limitar as possíveis perdas que
um investidor pode ter ao estar com seu patrimônio atrelado a
determinado ativo, por exemplo, para um acionista que possui ações de
determinada empresa se proteger contra uma possível queda no valor de
suas ações, ele pode comprar opções de venda at the money de suas ações
para que seu prejuízo máximo seja o prêmio.

\section{Travas}\label{travas}

Devido ao mercado de opções nos oferecer diversas possibilidades entre
call e put onde você pode estar comprado e/ou vendido irá surgir várias
posições a serem assumidas, para nos adequarmos ao quanto estamos
dispostos a encarar o risco parar atingirmos o retorno desejado. Essas
posições são conhecidas como ``Travas''.

Entendendo as travas, existem diversas estratégias, como por exemplo:
Trava de alta, trava de baixa, Long Straddle, Short Straddle, entre
outras. Mas afinal qual é o funcionamento delas? Supondo que o leitor
espere uma alta do mercado, no entanto acredite que não irá superar
determinado ponto ele poderá realizar uma Trava de alta. Onde comprará
uma opção de Call a um preço de strike X e vender outra Call com o preço
de strike Y, onde obrigatoriamente Y\textgreater{}X. Nesta operação
limitaremos o nosso ganho caso o mercado supere as nossas expectativas,
no entanto diminuiremos o custo da operação, o custo será o prêmio pago
pela opção X menos o prêmio recebido pela opção Y, para facilitar a
compreensão observemos o gráfico a seguir:

Nesse caso o valor do prêmio da compra foi de R\$30,00 enquanto a da
venda foi R\$10,00, assim limitamos nossa perda em R\$20,00, enquanto os
preços de strike da compra e da venda da call foram respectivamente
R\$250,00 e R\$300,00, fazendo o retorno máximo ser R\$30,00 que é a
diferença entre os valores de strike a serem realizados e descontado o
valor pago pelo prêmio.

Agora que o leitor já entendeu melhor o conceito da trava, vamos
explorar uma mais complexa a Long Butterfly. Aqui é realizado a compra
de uma put e call com preços de strike iguais, vendesse uma put com
preço inferior e vende alguma call com preço superior as iniciais.
Observe que pelo fato de contar com a venda de duas opções nessa
estratégia tem um custo de operação reduzido, no entanto o ideal é
utilizar em um mercado de pouca volatilidade, dado que se a volatilidade
ser alta perdesse a possibilidade de ter um ganho maior, nesse caso
recomenda o uso por exemplo de uma Long Straddle. Enfim vamos ao gráfico
para facilitar a compreensão da estratégia:

Teremos então a compra de uma put e call de strike iguais de R\$150,00 a
venda de uma put com strike inferior de R\$80, e a venda de uma call com
preço superior R\$220. Os valores exatos dos prêmios não nos interessam
no momento, porém é importante entender que teremos dois com saldos
positivos referente a nossa venda e dois negativos que advém das
compras, o resultado será nosso prejuízo máximo, olhando o gráfico nesse
caso é de R\$20,00. Pela área de retorno do gráfico podemos ver que
nosso risco está reduzido. Onde o pior cenário possível se encontra em o
preço do ativo-objeto se aproximar do valor do R\$150,00, que é onde os
contratos adquiridos não serão vantajosos em nenhuma ponta. No entanto
se o preço se aproximar de R\$220 poderemos exercer nosso direito da
compra da call inicial por R\$150,00(você terá o direito de comprar a um
preço inferior), o mesmo será valido caso haja uma queda do preço se
aproximado do valor de R\$80,00 onde a logica será a mesma só que aqui
será o usado o direito da compra da put por R\$150,00(Você terá o
direito de vender a um preço superior). Observe que apesar do nosso
risco ser reduzido, limita os nossos ganhos, com a venda da call e da
put com preços superior e inferior respectivamente.

Espero que o leitor tenha despertado interesse no assunto, com esse
conteúdo dominado já saberá o básico sobre opções, fique atento a novas
postagens em breve iremos mais a afundo explicando por exemplo o modelo
Black Scholes, como as opções são precificadas entre outros materiais.

\hypertarget{processos-estocasticos}{\chapter{Processos Estocásticos em
Finanças}\label{processos-estocasticos}}

Neste artigo abordaremos um assunto técnico, mas muito utilizado e de
fudamental importância para a precificação de instrumentos derivativos.
Será apresentado o conceito de processos estocásticos - PE, e sua
aplicação no mundo das finanças.

Um processo estocástico é a evolução temporal de uma determinada
variável de interesse que pode assumir valores aleatórios em cada ponto
no tempo. Em outras palavras, o caminho que a variável segue ao longo do
tempo evolui de maneira incerta. Estes processos podem se dar em tempo
discreto ou em tempo contínuo. Processo em tempo discreto são aqueles
onde o valor da variável pode se alterar somente em intervalos
pré-definidos de tempo, por exemplo ao final do dia. Em processos em
tempo contínuo, o valor de nossa variável está constantemente em
mudança, de forma aleatória seguindo alguma distribuição de
probabilidades.

Estes processos são muito importantes em finanças pois, é amplamente
aceito que a evolução do preço de ativos financeiros pode ser modelado
por um PE em tempo contínuo, sendo este modelo portanto, a base para a
teoria de precificação de ativos e da qual os derivativos fazem extenso
uso. Aprender sobre a evolução temporal do preço de uma ação através de
um processo estocástico é o primeiro passo para saber como atribuir um
preço a uma opção sobre esta ação, por exemplo.

Deve ser notado também que apesar de o preço dos ativos serem
\textbf{observados} apenas em intervalos discretos de tempo (apenas
quando existe transação) e assumirem valores também discretos (múltiplos
de um centavo), o preço e sua evolução estão ocorrendo continuamente,
nossas observações que são discretas. Desta forma os processos em tempo
contínuos são ideais para este tipo de modelagem.

\section{Processos de Markov}\label{markov}

Uma primeira definição de deve-se fazer para estudar PE aplicados a
evolução do preço de ações é o conceito de processo de Markov. Este tipo
de processo é tal que o histórico do processo que o levou até seu estado
atual, é \textbf{irrelevante} para a previsão de seu estado futuro. Ou
seja, toda a informação da história do processo já está contida no seu
valor atual. Quando consideramos que preços de ativos seguem um processo
de Markov, estamos assumindo válida pelo menos a forma fraca de mercados
eficientes.

Uma implicação desta suposição, verificada empiricamente, é que não se
pode obter lucros apenas seguindo padrões históricos do preço e
extrapolando-os no futuro. Outra, mais importante para nossos processos,
é que as distribuições de probabilidade que a variável aleatória segue
em cada ponto no tempo são \textbf{independentes}.

\section{Movimento Browniano}\label{mb}

Suponha um processo de Markov, que para fins de simplificação
consideraremos em tempo discretos. Se a distribuição de probabilidade
para o próximo incremento no valor do processo for uma Normal com média
zero e variância unitária, podemos representar este incremento por
\(\phi(0, 1)\). Como este é um processo de Markov, o segundo incremento
será independente do primeiro e terá novamente a mesma distribuição de
probabilidade. Qual seria então, a partir do período inicial até o
segundo período, a distribuição de probabilidade dos possíveis valores
de nosso hipotético processo? A reposta é a soma de duas normais
\(\phi(0, 1)\) que resulta em \(\phi(0, 2)\). Se assim continuarmos a
fazer previsões para T períodos a frente, nossa distribuição terá
densidade \(\phi(0, T)\).

Para tempos discretos, \(T\in\mathbb{Z}\) este é o processo do passeio
aleatório (\emph{Random Walk}), entretanto para tempo contínuo quando
\(T\in\mathbb{R}\) com incrementos acontecendo em intervalos de tempo
infinitesimalmente pequenos, este é o Movimento Browniano - MB, que
também é largamente conhecido como processo de Wiener.

\begin{Shaded}
\begin{Highlighting}[]
\NormalTok{t <-}\StringTok{ }\DecValTok{0}\OperatorTok{:}\DecValTok{500}
\NormalTok{m <-}\StringTok{ }\DecValTok{5} \CommentTok{# Numero de simulacoes}
\NormalTok{mc_names <-}\StringTok{ }\KeywordTok{paste0}\NormalTok{(}\StringTok{"Sim"}\NormalTok{, }\KeywordTok{seq_len}\NormalTok{(m)) }\CommentTok{# Nomes das simulacoes}
\NormalTok{sigma2 <-}\StringTok{ }\DecValTok{1}
\NormalTok{mu <-}\StringTok{ }\DecValTok{0}
\NormalTok{brown_t <-}\StringTok{ }\KeywordTok{matrix}\NormalTok{(}\DataTypeTok{nrow =} \KeywordTok{length}\NormalTok{(t), }\DataTypeTok{ncol =}\NormalTok{ m)}
\CommentTok{# MC Simulation}
\KeywordTok{set.seed}\NormalTok{(}\DecValTok{543210}\NormalTok{)}
\ControlFlowTok{for}\NormalTok{ (i }\ControlFlowTok{in} \KeywordTok{seq_len}\NormalTok{(m)) \{}
\NormalTok{  increments <-}\StringTok{ }\KeywordTok{rnorm}\NormalTok{(}\KeywordTok{length}\NormalTok{(t) }\OperatorTok{-}\StringTok{ }\DecValTok{1}\NormalTok{, mu, }\KeywordTok{sqrt}\NormalTok{(sigma2))}
\NormalTok{  brown_t[, i] <-}\StringTok{ }\KeywordTok{c}\NormalTok{(}\DecValTok{0}\NormalTok{, }\KeywordTok{cumsum}\NormalTok{(increments))}
\NormalTok{\}}

\KeywordTok{colnames}\NormalTok{(brown_t) <-}\StringTok{ }\NormalTok{mc_names}
\NormalTok{mb <-}\StringTok{ }\KeywordTok{as.tibble}\NormalTok{(}\KeywordTok{cbind}\NormalTok{(t, brown_t)) }\OperatorTok\StringTok{ }
\StringTok{  }\KeywordTok{gather}\NormalTok{(}\DataTypeTok{key =}\NormalTok{ sim, }\DataTypeTok{value =}\NormalTok{ value, }\OperatorTok{-}\NormalTok{t)}
\end{Highlighting}
\end{Shaded}

\begin{verbatim}
## Warning: `as.tibble()` is deprecated, use `as_tibble()` (but mind the new semantics).
## This warning is displayed once per session.
\end{verbatim}

\begin{Shaded}
\begin{Highlighting}[]
\KeywordTok{ggplot}\NormalTok{(mb, }\KeywordTok{aes}\NormalTok{(}\DataTypeTok{x =}\NormalTok{ t, }\DataTypeTok{y =}\NormalTok{ value, }\DataTypeTok{color =}\NormalTok{ sim)) }\OperatorTok{+}\StringTok{ }
\StringTok{  }\KeywordTok{geom_line}\NormalTok{() }\OperatorTok{+}
\CommentTok{#  geom_point(data = dens_tbl, aes(x = x, y = y), size = 1) +}
\StringTok{  }\KeywordTok{labs}\NormalTok{(}\DataTypeTok{title =} \StringTok{"5 realizações Movimento Browniano"}\NormalTok{,}
       \DataTypeTok{x =} \StringTok{"Tempo"}\NormalTok{,}
       \DataTypeTok{y =} \StringTok{"Valor"}\NormalTok{) }\OperatorTok{+}
\StringTok{  }\KeywordTok{guides}\NormalTok{(}\DataTypeTok{color =} \OtherTok{FALSE}\NormalTok{) }\OperatorTok{+}
\StringTok{  }\KeywordTok{scale_color_viridis_d}\NormalTok{() }\OperatorTok{+}
\StringTok{  }\KeywordTok{theme_economist_white}\NormalTok{()}
\end{Highlighting}
\end{Shaded}

\includegraphics{vol_surface_book_files/figure-latex/brownian_plot-1.pdf}

A figura acima mostra 5 realizações de um \emph{mesmo} processo
estocástico com média zero e variância unitária. É importante frisar que
o processo que gerou as cinco séries é exatamente o mesmo, sendo elas
tão distintas umas das outras ou não. Esta é uma importante
característica dos processos estocásticos nas aplicações reais, o que
nós observamos é apenas uma realização do processo, dentre as infinitas
possíveis.

\section{Definição}\label{definicao}

Agora que já foi passada a intuição sobre processos estocásticos,
pode-se partir para definições mais formais sobre estes processos. Vamos
adotar a notação do cálculo para tanto, e generalizar nosso MB
possibilitando-o que tenha média diferente de zero (\(\mu\)) e variância
qualquer (\(\sigma^2\)), mantendo estas constantes ao longo do tempo,
entretanto. Desta forma um movimento Browniano com deriva pode ser
descrito através da seguinte equação diferencial estocástica - EDE:

\begin{equation}
dX_t = \mu dt + \sigma dB_t
\label{eq:mb}
\end{equation}

onde \(dB_t\) é nosso MB padrão em um intervalo de tempo infinitesimal,
\(dt\).

O processo \(X_t\) possui uma taxa de deriva (média instantânea) igual a
\(\mu\) e volatilidade instantânea igual a \(\sigma\). Quando um PE
possui deriva igual a zero, como nosso MB padrão, o valor esperado deste
processo para qualquer período futuro será zero. Este fato deixa de ser
verdade no processo generalizado, com taxa de deriva diferente de zero.
Neste caso o processo evoluirá seguindo uma taxa crescente (se
\(\mu > 0\)) ou decrescente (se \(\mu < 0\)). Assim é possível, a partir
de um MB padrão, modelar outros PE que possuam tendência temporal e
variâncias diferentes.

\section{Movimento Browniano Geométrico}\label{mbg}

Apesar de o processo \(X_t\) ser bastante flexível e cobrir uma grande
gama de usos, ele ainda não é adequado para modelar o preço de ativos, e
isto se dá em função de o processo de Wiener, mesmo com deriva positiva,
poder atingir valores negativos com probabilidade maior que zero. Isto
implicaria na possibilidade do preço de uma ação ser negativo, algo que
obviamente não ocorre. Além desta impossibilidade, existe um outro
empecilho para se utilizar o MB para modelar o processo de preços, e
este é a deriva constante \(\mu\) com relação ao preço da ação.

A deriva pode ser interpretada como o valor esperado do retorno da ação
em um dado período de tempo. Este retorno esperado ele é pode ser
constante em termos percentuais (em um modelo simplificado), mas não em
termos absolutos! Ou seja, dependendo do preço da ação, R\$ 1,00 ou R\$
100,00, a deriva \(\mu\) deve ser diferente para que em termos
percentuais a relação seja constante.

A solução para estes dois problemas é modelar o preço como um processo
estocástico conhecido como Movimento Browniano Geométrico. Ele difere do
MB padrão pois assume que o \textbf{logaritmo} da variável aleatória
possui distribuição Normal. O MBG é a resolução para a seguinte EDE:

\begin{equation}
dX_t = \mu X_t dt + \sigma X_t dB_t
\label{eq:mbg}
\end{equation}

Veja que este é basicamente o mesmo processo MB, porém a deriva, termo
que multiplica \(dt\), varia linearmente com o valor do processo
(\(\mu X_t\)) assim como a volatilidade instantânea (\(\sigma X_t\)).

A solução para esta EDE, para um valor inicial qualquer de \(X\)
(\(X_0 > 0\)) é dada por:

\begin{equation}
X_t = X_0\exp\left(\left(\mu-\frac{\sigma^2}{2}\right)t+\sigma B_t\right)
\label{eq:mbgsol}
\end{equation}

A variável aleatória \(X\) segue um MB ao longo de uma trajetória
exponencial. É fácil verificar que, por ser exponencial, \(X_t\) nunca
terá valor negativo.

Esta é uma forma conveniente de representar a evolução de preços de um
ativo pois naturalmente surge o conceito de retornos logarítmos. O
log-retorno de \(X\) é dado por \(r_t=\ln(X_t/X_0)\) de onde inferimos
que se o processo de formação de preço de um ativo segue um MBG, então
seus log-retornos serão normalmente distribuídos com média
\(\mu-\frac{\sigma^2}{2}\) e volatilidade \(\sigma\) em uma unidade de
período considerado. Se escalarmos o período de tempo considerado para
\(T\), temos então que os retornos logarítmicos do ativo \(X\) seguem a
seguinte distribuição normal:

\begin{equation}
r_T \sim\phi\left(\left(\mu-\frac{\sigma^2}{2}\right)T, \sigma^2T\right)
\label{eq:rT}
\end{equation}

Abaixo apresentamos 5 realizações de um MBG com valor de deriva
\(\mu = 0,6\% a.p.\) e variância \(\sigma^2=1\% a.p.\).

\begin{Shaded}
\begin{Highlighting}[]
\NormalTok{t <-}\StringTok{ }\DecValTok{0}\OperatorTok{:}\DecValTok{500}
\NormalTok{m <-}\StringTok{ }\DecValTok{5} \CommentTok{# Numero de simulacoes}
\NormalTok{mc_names <-}\StringTok{ }\KeywordTok{paste0}\NormalTok{(}\StringTok{"Sim"}\NormalTok{, }\KeywordTok{seq_len}\NormalTok{(m)) }\CommentTok{# Nomes das simulacoes}
\NormalTok{sigma2 <-}\StringTok{ }\FloatTok{0.01}
\NormalTok{mu <-}\StringTok{ }\FloatTok{0.006} \CommentTok{# c(0.098, 0.099, 0.1, 0.101, 0.102) / 10}
\NormalTok{brown_t <-}\StringTok{ }\KeywordTok{matrix}\NormalTok{(}\DataTypeTok{nrow =} \KeywordTok{length}\NormalTok{(t), }\DataTypeTok{ncol =}\NormalTok{ m)}
\CommentTok{# MC Simulation}
\KeywordTok{set.seed}\NormalTok{(}\DecValTok{1234567}\NormalTok{)}
\ControlFlowTok{for}\NormalTok{ (i }\ControlFlowTok{in} \KeywordTok{seq_len}\NormalTok{(m)) \{}
\NormalTok{  log_ret <-}\StringTok{ }\KeywordTok{rnorm}\NormalTok{(}\KeywordTok{length}\NormalTok{(t) }\OperatorTok{-}\StringTok{ }\DecValTok{1}\NormalTok{, mu }\OperatorTok{-}\StringTok{ }\NormalTok{(sigma2 }\OperatorTok{/}\StringTok{ }\DecValTok{2}\NormalTok{), }\KeywordTok{sqrt}\NormalTok{(sigma2))}
\NormalTok{  brown_t[, i] <-}\StringTok{ }\KeywordTok{c}\NormalTok{(}\DecValTok{1}\NormalTok{, }\KeywordTok{cumprod}\NormalTok{(}\KeywordTok{exp}\NormalTok{(log_ret)))}
\NormalTok{\}}

\KeywordTok{colnames}\NormalTok{(brown_t) <-}\StringTok{ }\NormalTok{mc_names}
\NormalTok{mb <-}\StringTok{ }\KeywordTok{as.tibble}\NormalTok{(}\KeywordTok{cbind}\NormalTok{(t, brown_t)) }\OperatorTok\StringTok{ }
\StringTok{  }\KeywordTok{gather}\NormalTok{(}\DataTypeTok{key =}\NormalTok{ sim, }\DataTypeTok{value =}\NormalTok{ value, }\OperatorTok{-}\NormalTok{t)}
  
\KeywordTok{ggplot}\NormalTok{(mb, }\KeywordTok{aes}\NormalTok{(}\DataTypeTok{x =}\NormalTok{ t, }\DataTypeTok{y =}\NormalTok{ value, }\DataTypeTok{color =}\NormalTok{ sim)) }\OperatorTok{+}\StringTok{ }
\StringTok{  }\KeywordTok{geom_line}\NormalTok{() }\OperatorTok{+}
\StringTok{  }\KeywordTok{labs}\NormalTok{(}\DataTypeTok{title =} \StringTok{"5 realizações Movimento Browniano Geométrico"}\NormalTok{,}
       \DataTypeTok{x =} \StringTok{"Tempo"}\NormalTok{,}
       \DataTypeTok{y =} \StringTok{"Valor"}\NormalTok{) }\OperatorTok{+}
\StringTok{  }\KeywordTok{guides}\NormalTok{(}\DataTypeTok{color =} \OtherTok{FALSE}\NormalTok{) }\OperatorTok{+}
\StringTok{  }\KeywordTok{scale_y_continuous}\NormalTok{(}\DataTypeTok{breaks =} \DecValTok{0}\OperatorTok{:}\DecValTok{10}\NormalTok{) }\OperatorTok{+}
\StringTok{  }\KeywordTok{scale_color_viridis_d}\NormalTok{() }\OperatorTok{+}
\StringTok{  }\KeywordTok{theme_economist_white}\NormalTok{()}
\end{Highlighting}
\end{Shaded}

\includegraphics{vol_surface_book_files/figure-latex/mbg_plot-1.pdf}

O Movimento Browniano Geométrico aqui demonstrado serve de base para o
famoso modelo \textbf{Black \& Scholes} de precificação de opções, o
qual assume que o ativo subjacente à opção (por exemplo, a ação de uma
empresa) tem seu preço formado por um processo MBG.

\hypertarget{bsm}{\chapter{Modelo Black-Scholes-Merton}\label{bsm}}

Neste artigo desenvolveremos o modelo para precificação de opções do
tipo europeias proposto \citet{Black1973} e posteriormente expandido por
\citet{Merton1976}. A derivação deste modelo se baseia nos conceitos
apresentados sobre processos estocásticos do Capítulo
\ref{processos-estocasticos}.

Antes de entrar na modelagem desenvolvida pelos autores acima citados,
iremos tratar de algumas definições essenciais, como por exemplo a
precificação de ativos em um mundo neutro ao risco (\emph{risk neutral
valuation}) e portfólio de replicação (\emph{replicating portfolio}),
que são largamente utilizados na precificação de quaisquer derivativos,
e não somente opções.

\section{Portfólio de replicação}\label{portfolio-de-replicacao}

Suponha que se deseja precificar uma opção de compra sobre uma ação,
vamos denotar o preço desta \emph{call} de \(f_t\). Sabemos que na data
de expiração, \(T\), da opção de compra seu preço será:
\(f_T=max(S_T - K, 0)\), onde \(S_T\) é o preço da ação subjacente na
data \(T\) e \(K\) é o preço de exercício da opção. Podemos criar um
portfólio que envolva um ativo livre de risco, com preço \(B_t\) e a
ação objeto do derivativo, \(S_t\), que recrie o mesmo valor de
pagamento que a opção na data de expiração. Ou seja, criamos o portfólio
de replicação \(P_t=\Delta S_t + b B_t\), no qual devemos escolher os
valores de \(\Delta\) e \(b\) de tal forma que \(f_T = P_T\).

De fato, para apenas um período a frente, podemos tomar emprestado a
taxa de juros livre de risco\footnote{Esta é uma suposição do modelo,
  pode não ser verdade para algum investidor qualquer mas se for para
  algum outro investidor representativo, o princípio de ausência de
  arbitragem passa a valer, uma vez que este segundo investidor
  explorará o mercado e levará o preço do derivativo para o resultado
  requerido.}, \(r\) um valor igual ao preço corrente da ação, \(S_0\) e
comprá-la, ao mesmo tempo em que se ``trava'' (faz \emph{hedge}) deste
portfólio vendendo a opção. O lucro desta operação em \(T\) deve ser
zero, pois é um portfólio travado. Com estas premissas é possível
calcular o preço do prêmio da opção a ser cobrado no período inicial
para que o valor esperado da operação como um todo seja zero.

Este portfólio formado pela venda do ativo livre de risco e compra do
ativo objeto é denominado \emph{portfólio de replicação}, pois, ele
fornece um fluxo de caixa igual ao derivativo que buscamos precificar.
Ao adicionarmos ao portfólio de replicação a venda (ou seja, o negativo)
do derivativo, temos um portfólio \emph{hedgeado}, onde não existe mais
incerteza com relação ao seu retorno esperado.

\section{Precificação neutra ao
risco}\label{precificacao-neutra-ao-risco}

A metodologia de precificação de derivativos dentro do \emph{mundo
neutro ao risco}\footnote{A precificação dentro do mundo neutro ao risco
  mereceria um, ou mais, \emph{posts} por si só. Aqui lançaremos apenas
  os principais resultados que nos permitem encontrar o modelo de B\&S.}
é o carro-chefe das metodologias para se avaliar estes instrumentos. O
princípio de precificação neutra ao risco afirma que um derivativo pode
ser valorado através das seguintes suposições:

\begin{itemize}
\tightlist
\item
  o retorno esperado do ativo objeto é a taxa livre de risco, e
\item
  o valor esperado do derivativo na expiração deve ser descontado pela
  taxa livre de risco para trazê-lo a valor presente.
\end{itemize}

É claro que o mundo real não é neutro ao risco, entretanto uma das
provas que a metodologia faz é que, ao precificarmos um derivativo de
forma relativa ao preço do ativo objeto, a precificação neutra ao risco
encontra o mesmo valor para o derivativo que uma metodologia que leve em
conta as preferências ao risco dos investidores. Entretanto, precificar
um derivativo assumindo neutralidade ao risco é muito mais simples que
encarar um modelo baseado no mundo real.

De fato a neutralidade ao risco soa um tanto quanto estranha a primeira
vez. Porém, ela tem uma explicação lógica de sua validade nas
circunstâncias em que é desenvolvida. Se os investidores forem avessos
ao risco, por exemplo, os retornos esperados para o ativo objeto terão
embutidos um prêmio pelo risco. Acontece que este prêmio pelo risco
também deverá estar na taxa de desconto do valor esperado de pagamento
do derivativo, de forma que este prêmio ao risco é cancelado.

É comum na literatura de derivativos encontrarmos os termos mundo-P e
mundo-Q. O mundo-P se refere ao mundo real, com probabilidades P reais
de ocorrência de eventos, equanto que o mundo-Q é o mundo neutro ao
risco, onde as probabilidades Q são ajustadas (tecnicamente suas medidas
são alteradas) para refletir esta neutralidade. No mundo-Q é comum
denotarmos o valor esperado de alguma variável aleatória com a seguinte
notação: \(\mathbb{E_Q}[\cdot]\).

Assim, a precificação de derivativos supondo um mundo neutro ao risco
chega no preço correto para todos os mundos.

\section{Encontrando a equação de
Black-Scholes}\label{encontrando-a-equacao-de-black-scholes}

Vamos partir do princípio que nossa ação, objeto da opção que desejamos
precificar, siga um MBG conforme descrito no
\protect\hyperlink{processos-estocasticos}{capítulo anterior}. Portanto,
o preço de nossa ação no período \(t\) deve observar a seguinte equação
diferencial estocástica:

\begin{equation}
dS_t=\mu S_t+\sigma S_t dW_t 
\label{eq:ds}
\end{equation}

onde \(dW_t\) é o movimento Browniano, ou processo de Wiener.

Esta equação resume as principais hipóteses do modelo Black\&Scholes de
precificação de opções, são elas:

\begin{itemize}
\tightlist
\item
  O preço do ativo objeto é um processo estocástico e segue uma
  distribuição log-normal;
\item
  O retorno esperado (\(\mu\)) e a volatilidade (\(\sigma\)) deste ativo
  são \textbf{constantes}, tanto no tempo quanto com relação ao próprio
  nível de preço do ativo objeto.
\end{itemize}

Ademais destas hipóteses, temos aquelas relacionadas a racionalidade dos
mercados e ao princípio de ausência de oportunidades de arbitragem.
Estas hipóteses nos levam a validade do mundo neutro ao risco e
portanto, a resolução do modelo da forma como descreveremos abaixo.

Nossa opção será descrita por um portfólio de replicação, e se tomarmos
seu preço no instante \(t\), então a opção também deve seguir uma
equação diferencial estocástica da forma:

\begin{equation}
df_t=\Delta dS_t+b\,dB_t 
\label{eq:df1}
\end{equation}

aqui \(dB_t\) representa a variação do ativo livre de risco (dinheiro)
dentro de um período de tempo \(dt\). Sabemos que o valor do ativo livre
de risco não envolve incerteza alguma, é determinístico, e seu
rendimento é a taxa de juros livre de risco. Assim, para uma taxa
continuamente composta \(r\) uma unidade de \(B\) evolui através de
\(B_t=e^{rt}\), logo:

\begin{equation}
dB_t=rBdt
\label{eq:db}
\end{equation}

De acordo com o
\href{https://pt.wikipedia.org/wiki/Lema_de_It\%C5\%8D}{lema de Ito} o
diferencial de uma função que dependa do tempo e de um processo
estocástico pode ser encontrado da seguinte forma:

\begin{equation}
df_t=\frac{\partial f}{\partial t}dt + \frac{\partial f}{\partial S}dS + \frac{1}{2}\frac{\partial^2 f}{\partial S^2}dS^2
\label{eq:df2}
\end{equation}

onde \(dS^2\) é a variação quadrática de nosso processo \(S_t\) que é um
movimento browniano geométrico, logo \(dS^2=\sigma^2S^2dt\).

Igualando as equações \eqref{eq:df1} e \eqref{eq:df2}, fazendo as devidas
substituições trazidas pelas equações \eqref{eq:ds} e \eqref{eq:db} e por
fim rearranjando os termos, chegamos na seguinte relação:

\begin{equation}
\left( \frac{\partial f}{\partial t} + \frac{\partial f}{\partial S}\mu S + \frac{1}{2}\frac{\partial^2 f}{\partial S^2}\sigma^2 S^2 - \Delta\mu S - rbB \right)dt + \left( \frac{\partial f}{\partial S}\sigma S - \Delta \sigma S \right)dW = 0 
\end{equation}

que para ser válida para todo \(t\), cada termo entre parênteses deve
ser igual a zero simultaneamente. Rapidamente chegamos aos valores
necessários de nosso portfólio de replicação.

\begin{equation}
\Delta = \frac{\partial f}{\partial S} 
\label{eq:delta}
\end{equation}

\begin{equation}
rbB = \frac{\partial f}{\partial t}+\frac{1}{2}\frac{\partial^2 f}{\partial S^2}\sigma^2S^2 
\label{eq:rbb1}
\end{equation}

Através da equação \eqref{eq:df1}, integrando-a, chegamos na relação
\(bB = f - \Delta S\) e pré-multiplicando ambos os lados por \(r\) temos
então que:

\begin{equation}
rbB = r(f-\Delta S) 
\label{eq:rbb2}
\end{equation}

Finalmente, igualando as equações \eqref{eq:rbb1} e \eqref{eq:rbb2} e
substituindo o valor de \(\Delta\) encontramos a famigerada equação
diferencial parcial - EDP - de Black\&Scholes:

\begin{equation}
\frac{\partial f}{\partial t}+\frac{1}{2}\sigma^2S^2\frac{\partial^2 f}{\partial S^2}+rS\frac{\partial f}{\partial S} - rf = 0
\label{eq:BS}
\end{equation}

\section{Solução analítica}\label{solucao-analitica}

A equação possui diferentes formas de resolução\footnote{Uma
  demonstração completa de como encontrar a solução para a EDP de
  Black\&Schole pode ser encontrada em
  \url{https://planetmath.org/AnalyticSolutionOfBlackScholesPDE}}, a
precificação do derivativo irá depender da forma que resolvermos a
equação \eqref{eq:BS}. O Modelo Black\&Scholes é famoso por conseguir
precificar opções call e put europeias, onde a resolução da equação fará
uso dos \emph{payoffs}. Para uma call:

\begin{equation}
\displaystyle f(S,T)=\max(S-K,0)
\label{eq:cpayoff}
\end{equation}

Já no caso de uma put:

\begin{equation}
\displaystyle f(S,T)=\max(K-S,0)
\label{eq:ppayoff}
\end{equation}

Onde:

\begin{itemize}
\tightlist
\item
  T é a data de vencimento da opção,
\item
  K é o preço de exercício, e
\item
  S é o preço do ativo subjacente.
\end{itemize}

Utilizando os \emph{payoffs} dados nas equações \eqref{eq:cpayoff} e
\eqref{eq:ppayoff}, e resolvendo a EDP de Black\&Scholes \eqref{eq:BS},
iremos obter o modelo para se precificar os derivativos citados.

Para uma Call temos que:

\begin{equation}
C(S,t)=SN(d_{1})-Ke^{-r(T-t)}N(d_{2})
\label{eq:call}
\end{equation}

Já para um Put chegamos a:

\begin{equation}
P(S,t)=Ke^{-r(T-t)}N(-d_{2})-SN(-d_{1})
\label{eq:put}
\end{equation}

onde:

\begin{equation}d_{1}={\frac {\ln(S/K)+(r+\sigma ^{2}/2)(T-t)}{\sigma {\sqrt {T-t}}}}
\label{eq:d1}
\end{equation}

\begin{equation}d_{2}={\frac {\ln(S/K)+(r-\sigma ^{2}/2)(T-t)}{\sigma {\sqrt {T-t}}}}=d_1-\sigma\sqrt{T-t}
\label{eq:d2}
\end{equation}

É a partir das equações \eqref{eq:call} e \eqref{eq:put} que se obtém as
chamadas gregas, que são as sensibilidades do preço do derivativo em
relação a alterações nos parâmetros do modelo. Explicaremos as gregas em
maiores detalhes mais adiante.

\section{Paridade compra e venda}\label{putcallparity}

Agora vamos falar um pouco sobre a paridade entre opções de compra e
venda, algo que nos ajuda a precificar algum derivativo quando já
conhecemos o preço de outro derivativo com especificações semelhantes.

Assumindo ausência de oportunidade de arbitragem, com um ativo
subjacente que tenha liquidez, iremos verificar a paridade Call-Put que
define uma relação entre os preços de uma call e put do tipo europeu,
desde que possuam o \textbf{mesmo tempo de maturidade} e \textbf{preço
de strike}. Aqui possuiremos menos premissas que o modelo Black\&Scholes
e premissas mais simples, tal relação poderá ser utilizada para
encontrar o valor justo de uma opção. Com a ausência de arbitragem
observe que dois portfolios que sempre geram o mesmo payoff em um
instante T devem ter o mesmo valor em qualquer instante intermediário.

Imagine um primeiro portfólio onde o investidor compre uma opção de
compra C, a qual possui tempo de maturidade T e preço de strike K, sobre
algum ativo subjacente que chamaremos de S e compre um título B que seu
valor no período T seja de {\$}30. E um segundo portfolio onde tenha o
próprio ativo subjacente S que esteja sendo negociado a {\$}25 e compre
uma put P com um preço de strike K e maturidade T. Estes são porfólios
de replicação entre si, eles terão exatamente o mesmo \emph{payoff} no
período T, e na ausência de arbitragem, podemos calcular seus retornos
através da precificação neutra ao risco. Teremos uma relação onde os
valores dos nossos portfolios serão:

\begin{equation}
\mathbb{E_Q}[S_t + P_t] = \mathbb{E_Q}[C_t + B_t]; \quad t \leq T
\label{eq:parity}
\end{equation}

Como estamos em um mundo neutro ao risco, \(P_t\) e \(C_t\) são os
preços da put e da call dados pelo modelo B\&S, enquanto
\(\mathbb{E_Q}[B_t]\) se resume a \(Ke^{-r(T-t)}\), ou seja, a posição
atual em ativo livre de risco deve ser o valor presente do strike das
opções. O preço do subjacente é o próprio preço atualmente observado.

Como \(S_t\) e \(B_t=Ke^{-r(T-t)}\) são conhecidos, se no mercado
existir um preço \(C_t\) então podemos calcular \(P_t\) e vice-versa.
Esta é a essência da paridade compra e venda.

Observe que se essa relação não for mantida, teremos arbitragem:
Suponhamos que \(S_t > B_t\) e que a put esteja com um premio mais alto
que a call no entanto ambos possuem um preço de strike {\$}30 e
maturidade T, que os prêmios sejam {\$}20 e {\$}15 respectivamente
teremos então: 30 + 20 = 15 + 25, nesse caso teremos 50 \(\neq\) 40,
então vende-se o que está mais caro, no caso a ação e a put e compraria
o título e a call, chegando ao um lucro de 50-40=10, de {\$}10 sem
risco. Observe que independentemente de a ação subir acima ou cair
abaixo do strike, o lucro obtido nesta operação será sempre de {\$}10.

\section{As Gregas}\label{gregas}

As letras gregas utilizadas no mercado de opções são usadas para denotar
as sensibilidades do preço da opção com relação a variação de alguma das
variáveis do modelo, a seguir entraremos em detalhes sobre as principais
gregas usadas para a análise de opções.

\subsection{Delta}\label{delta}

Já havíamos definido o delta anteriormente ao encontrarmos a EDP de
Black\&Scholes, isto é:

\begin{equation}
\Delta = \frac{\partial f}{\partial S}
\label{eq:delta}
\end{equation}

O delta mede a taxa com que o preço da opção muda conforme o valor do
ativo subjacente oscila, seu valor pode variar entre 1 e 0 para call e
entre 0 e -1 para put.

Exemplo: imagine uma opção de compra de 100 ações que possui um delta de
0,72, caso o ativo subjacente aumente {\$}1 o valor da opção aumentará
{\$}72 (pois representa 100 ações), e caso o valor do ativo subjacente
diminua o preço da opção diminuirá {\$}72.

O mesmo acontece com opções de venda, só que de uma forma inversamente
proporcional pelo seu delta ser negativo. Uma opção de venda de 100
ações que possui um delta de -0,40 vai valer menos {\$}40 caso o preço
do seu ativo subjacente aumente {\$}1, e vai valer mais {\$}40 caso o
preço do seu ativo subjacente diminua {\$}1.

O strike da opção influência o delta diretamente, quanto mais
in-the-money for uma opção, call ou put, maior será o seu delta em valor
absoluto, e quanto mais out-the-money, menor será o módulo de seu delta.

\subsection{Gamma}\label{gamma}

\begin{equation}
\Gamma = \frac{\partial \Delta}{\partial S} = \frac{\partial^2 f}{\partial S^2}
\label{eq:gamma}
\end{equation}

O gamma mede a taxa com que o delta muda a cada oscilação do preço do
ativo subjacente, seu impacto aumenta conforme o valor atual do ativo se
aproxima do strike, sendo responsável pela convexidade do valor da
opção. Opções de alto gamma são chamadas de explosivas, uma vez que
mesmo pequenas variações no preço do ativo se traduzem em grandes
oscilações no preço da opção.

\subsection{Theta}\label{theta}

\begin{equation}
\Theta = -\frac{\partial f}{\partial \tau}, \quad \tau=T-t
\label{eq:theta}
\end{equation}

O Theta é uma taxa que mede o efeito do tempo sobre o preço da opção.
Como as opções tem um maior valor de acordo com a quantidade de tempo
até a data de vencimento, seu valor também diminui conforme o tempo
passa, theta é a letra que mede essa variação (que sempre é negativa). O
valor de theta representa a quantidade de dinheiro perdida no prêmio da
opção a cada dia que passa.

\subsection{Vega}\label{vega}

\begin{equation}
\displaystyle {\mathcal {V}}=\frac{\partial f}{\partial \sigma} 
\label{eq:vega}
\end{equation}

O vega é uma taxa que mede o efeito da mudança da volatilidade no preço
da opção. Seu valor é praticamente constante em opções com a mesma data
de vencimento, esse valor aumenta em datas de vencimento mais distantes
devido ao maior espaço de tempo em que a volatilidade atua, devido ao
maior intervalo de tempo que há para ocorrer mudanças nos preços.

\subsection{Rho}\label{rho}

\begin{equation}
\displaystyle \rho =\frac{\partial f}{\partial r}
\label{eq:rho}
\end{equation}

O rho é uma taxa que mede a sensitividade do preço da ação em relação à
taxa livre de risco, caso o valor de de rho determinada opção seja 0,7,
para cada aumento de 1\% da taxa livre de risco o valor da opção
aumentará 0,7\%. Essa taxa influi principalmente no preço de opções com
uma data de vencimento extremamente distante, não afetando muito o preço
de opções cuja data de vencimento é próxima.

\chapter{Smile de Volatilidade}\label{smile}

A volatilidade instantânea, \(\sigma\), do ativo subjacente é a única
variável no modelo B\&S que não pode ser diretamente observada. De fato,
a volatilidade (ou equivalentemente a variância) de um ativo é dita uma
variável \textbf{latente}. Sabemos que ela existe e possui algum valor
no processo gerador, o processo pelo qual os preços são formados, porém
não conseguimos observá-la diretamente, apenas estimá-la. Uma das formas
de estimação de volatilidade pode ser a partir de dados históricos, mas
várias outras formas existem, entre elas processos GARCH, volatilidade
realizada, volatilidade estocástica, etc.

Uma vez que a volatilidade não pode ser diretamente observada, a prática
comum no mercado é fazer o caminho inverso. Considerar os preços de
mercado para as opções como dado, e a partir do modelo
\protect\hyperlink{bsm}{B\&S} inverter a equação de preço da Call ou Put
para encontrar a volatilidade deste modelo que é compatível com os
preços de mercado. A esta volatilidade encontrada damos o nome de
\textbf{volatilidade implícita}.

Portanto, o smile de volatilidade que tratamos neste post é na verdade
um gráfico entre a volatilidade implícita, retirada de opções Européias
(baunilhas, do inglês vanilla options) a partir do modelo B\&S, contra
os \emph{strikes} destas opções.

\section{Reparametrizando B\&S e definição de
moneyness}\label{reparametrizando}

Nem sempre é interessante plotar o smile contra os \emph{strikes}
propriamente ditos, uma forma de avaliar o quanto uma opção está
``dentro, fora ou no dinheiro'' pode ser a grega Delta ou então o
chamado \emph{moneyness} (por favor, se alguém tiver uma boa tradução
para este termo, deixe nos comentários). Tradicionalmente a medida de
\emph{moneyness} é a relação \(K/S\), ou seja o strike contra o preço
corrente do subjacente. Porém existem outras definições mais
interessantes para se trabalhar, entretanto, antes devemos fazer uso de
algumas definições e vamos reparametrizar as expressões \(d1\) e \(d2\)
do modelo \protect\hyperlink{bsm}{B\&S}.

Lembrando que em precificação de opções estamos no mundo neutro ao
risco, vamos definir o valor \emph{forward}, \(F\) do subjacente como o
valor corrente composto pela taxa livre de risco até a maturidade da
opção, ou seja:

\begin{equation}
F=e^{r\tau}S
\end{equation}

A \textbf{volatilidade (implícita) total} pode ser definida como a
volatiliade reescalada pela raiz do tempo, que nos dá uma informação da
volatiliade esperada para o subjacente do período corrente até a
maturidade. Da mesma forma, a \textbf{variância total}. Denotanto a
volatilidade total por \(\theta\) e a variância total por \(w\), temos:

\begin{equation}
\theta=\sigma_{imp}\cdot \sqrt{\tau} 
\label{eq:voltotal}
\end{equation}

e

\begin{equation}
w=\sigma_{imp}^2\cdot\tau
\label{eq:vartotal}
\end{equation}

Vamos também definir a medida \emph{forward log-moneyness} e denotá-la
por \(k\). Esta será a medida de \emph{moneyness} que iremos utilizar ao
longo deste e de outros artigo, portanto iremos utilizar este termo para
designar o \emph{forward log-moneyness} a não ser que expresso de forma
contrária no texto.

\begin{equation}
k=\ln\left(\frac{K}{S}\right)-r\tau=\ln\left(\frac{K}{F}\right)
\label{eq:flmoneyness}
\end{equation}

Logo, o \emph{strike} está relacionado ao \emph{moneyness} de forma que:
\(K=Fe^k\).

Podemos agora reparametrizar as expressões \(d1\) e \(d2\) do modelo
B\&S de forma que serão mais facilmente trabalhadas em modelos de
volatilidade. Lembrando destas expressões que já foram apresentadas no
Capítulo \ref{bsm}:

\begin{align}
&d_{1}={\frac {\ln(S/K)+(r+\sigma ^{2}/2)(\tau)}{\sigma {\sqrt {\tau}}}}\\
&d_{2}={\frac {\ln(S/K)+(r-\sigma ^{2}/2)(\tau)}{\sigma {\sqrt {\tau}}}}=d_1-\sigma\sqrt{\tau}
\end{align}

Substituindo as expressões para \emph{forward log-moneyness} e
volatilidade total nas definições acima temos as novas parametrizações
para \(d1\) e \(d2\):

\begin{equation}
d_{1}={-\frac{k}{\theta}+\frac{\theta}{2}}
\label{eq:d1mod}
\end{equation}

\begin{equation}
d_{2}={-\frac{k}{\theta}-\frac{\theta}{2}}=d_1-\theta
\label{eq:d2mod}
\end{equation}

Retomando o valor da opção do tipo Call no modelo B\&S, podemos
reescrever sua fórmula de apreçamento da seguinte forma:

\begin{equation}
\begin{aligned}
C(K, \tau)=&SN(d1)-Ke^{-r\tau}N(d2)\\
e^{r\tau}C(K, \tau)=&FN(d1)-KN(d2)\\
C_B=&F\left[N(d1)-e^kN(d2)\right]
\end{aligned}
\label{eq:cblack}
\end{equation}

Esta equação é conhecida como a forma de Black de precificação
\emph{(Black Call price formula)}, que relaciona os valores
\emph{forward} da opção (também conhecido como valor não descontado), do
subjacente e do \emph{strike}. Esta formulação é particularmente útil
quando formos extrair a distribuição neutra ao risco do subjacente que
está implícita nos preços de mercado das opções.

\section{Características de smiles de volatilidade}\label{caracsmile}

Caso o modelo de Black, Schole e Merton estivesse em acordo com a
realidade, e os ativos tivessem seus preços formados a partir de um
verdadeiro MBG, a volatilidade implícita seria uma constante. O gráfico
do smile de volatilidade seria uma reta horizontal, com a mesma
volatilidade para qualquer nível de \emph{moneyness} e se considerarmos
a superfície toda (que leva em conta os diversos tempos para expiração)
esta seria paralela ao domínio \((k, \tau)\). Não estaríamos escrevendo
(e você lendo) este artigo se este fosse o caso\ldots{}

O fato é que o modelo B\&S é um modelo muito restritivo, com inúmeras
suposições que não se verificam no mundo real e que por conseguinte,
tornam os resultados do modelo pouco acurados. Entretanto este é um
modelo muito conhecido, de fácil assimilação por parte dos agentes de
mercado e que virou a língua franca nos mercados de derivativos. Se
todos os \emph{traders} conversarem em termos do modelo B\&S, todos se
entenderão, mesmo que internamente cada um possua seu próprio modelo de
apreçamento.

Entre as características tipicamente observadas em smiles (e
superfícies) de volatilidades pode-se citar:

\begin{itemize}
\tightlist
\item
  As volatilidades implícitas variam conforme o \emph{strike} e prazo de
  expiração
\item
  Smiles apresentam \emph{skew}. Maior inclinação em uma das asas,
  representando uma maior probabilidade daqueles \emph{strikes}
  acontecerem
\item
  Smiles de equity tipicamente são negativos
\item
  Mercados diferentes apresentam padrões de smile diferentes
\end{itemize}

\subsection{Mercados cambiais}\label{mercados-cambiais}

Opções sobre moedas possuem tipicamente um smile de volatilidade
conforme mostrado na figura \ref{fig:smile-cambial} abaixo. A
volatilidade implícita é relativamente baixa para opções ATM. Esta
torna-se progressivamente maior quando a opção se move para dentro do
dinheiro ou para fora.

\begin{figure}
\centering
\includegraphics{img/smile_cambial.png}
\caption{\label{fig:smile-cambial}Smile típico de um mercado cambial.}
\end{figure}

Caso a distribuição dos preços do ativo subjacente, neste caso uma taxa
de câmbio fosse perfeitamente log-normal como no modelo B\&S, o smile
não teria esta curvatura. Desta forma podemos afirmar que o mercado, ao
precificar as opções, acredita que a distribuição deste ativo possui
caudas com maior densidade que supõe a log-normal, existem maiores
probabilidades de retornos muito baixos ou muito altos.

\subsection{\texorpdfstring{Mercados de
\emph{equities}}{Mercados de equities}}\label{mercados-de-equities}

Nos mercados de \emph{equities}, ações, índices de ações e ETFs, por
exemplo, o smile apresenta uma característica de assimetria (skew, em
inglês) negativa. A asa esquerda (parte onde as puts estão fora do
dinheiro) apresenta valores de volatilidade implícita muito maiores que
suas contrapartes no lado das calls. Este comportamento reflete a
percepção de mercado de uma maior probabilidade de grandes perdas nas
ações que altos ganhos, gerando portanto, uma distribuição de preços
assimétrica. Como existe uma maior probabilidade de perdas extremas, o
seguro para estas, ou seja, uma put é relativamente mais cara que uma
call.

\begin{figure}
\centering
\includegraphics{img/smile_equities.png}
\caption{\label{fig:smile-equities}Smile típico de uma ação ou índice de
ações.}
\end{figure}

\section{Smile como forma de precificação}\label{smileprecificacao}

Analisando a equação de B\&S com a parametrização para \(d1\) e \(d2\)
dada no início deste artigo é possível verificar que existe uma relação
direta entre volatilidade implícita e preço de uma opção, seja esta uma
call ou put.

Como \(d1\) é estritamente crescente em \(\theta\) e \(d2\) é
estritamente decrescente e ao mesmo tempo o preço de uma opção é
crescente em d1 e decrescente em d2, logo, temos uma relação direta
entre o preço de uma opção e sua volatilidade implícita para uma dada
maturidade. Em outras palavras, em um smile, tudo o mais constante,
quanto maior a volatilidade implícita maior o preço da opção naquele
\emph{strike}.

Outra forma de verificar esta relação é perceber que a grega Vega, que é
calculada da mesma forma para calls e puts, é sempre positiva. Ou seja,
um aumento no valor da volatiliade sempre leva a elevações no preço de
uma opção.

Desta forma é normal entre os praticantes de mercado fazer a
precificação de opções em termos de ``pontos de volatilidade'' e não em
valores monetários propriamente ditos. Isto porque o modelo B\&S, apesar
de não ser o modelo correto (nenhum é) para a precificação de opções, é
conhecido e de fácil entendimento para todos. Então todos os praticantes
podem fazer suas cotações em termos de volatilidades implícitas, que são
extraídas de opções baunilhas com o modelo B\&S, e somente na hora de
fechar um negócio e liquidar o pagamento, o preço efetivo a ser pago é
acordado entre as partes.

\section{Estrutura a termo}\label{estrutura-a-termo}

O mercado precifica a volatilidade implícita de forma que esta dependa
também do tempo até expiração, bem como do preço de exercício, agregando
uma segunda dimensão ao smile e transformando-o na famigerada
\textbf{superfície de volatilidade implícita}.

A volatilidade implícita tende a ser uma função crescente da maturidade
quando as volatilidades de curto prazo são historicamente baixas e
função decrescente da maturidade quando as volatilidades de curto prazo
são historicamente altas. Isso porque existe uma expectativa de reversão
a uma média de longo prazo embutida na volatilidade. Esta característica
é explorada explicitamente por alguns modelos de volatilidade, como em
\citet{Heston1993}.

As superfícies de volatilidade combinam smiles com a estrutura a termo
de volatilidade para tabular valores apropriados para precificar uma
opção com qualquer preço de exercício e prazo de expiração.

Da mesma forma como a curva de juros em um dado momento é uma descrição
concisa dos preços dos títulos negociados naquele mercado, assim, para
um ativo subjacente em particular em determinado momento, a superfície
de volatilidade implícita fornece uma descrição resumida de seu mercado
de opções. Considerando que os rendimentos dos títulos são diferenciados
pelo seu tempo até o vencimento, as opções são diferenciadas por seu
tempo até a expiração e o \emph{strike}, logo requerem uma superfície ao
invés de uma curva.

A figura \ref{fig:superficie} demonstra uma superfície de volatilidade
implícita do \texttt{SPX} em 15/09/2005, conforme apresentado em
\citet{Gatheral2011}.

\begin{figure}
\centering
\includegraphics{img/spx_vol_surface.png}
\caption{\label{fig:superficie}Superfície de volatilidade implícita.}
\end{figure}

\section{Arbitragem estática}\label{arbestatica}

Antes de definir o que é arbitragem estática que pode estar presente em
uma superfície de volatilidade (ou na superfície de preço de opções),
vamos partir para a intuição por trás desta definição.

O princípio de ausência de arbitragem é dominante na teoria financeira.
Este princípio nos informa que não deve existir lucro sem que se incorra
em algum tipo de risco, o lucro sempre é a remuneração do investidor que
aceitou carregar alguma forma de risco durante o investimento. Portanto,
não devem existir perfis de lucro acima da taxa livre de risco
(\emph{payoffs} positivos) com probabilidade de 100\%.

Primeiro consideramos uma trava de alta com opções do tipo call.
Excluindo-se os custos de transação, esta operação sempre oferece um
retorno positivo ou zero, conforme a figura \ref{fig:trava-alta}. Por
mais que esta estratégia esteja montada fora do dinheiro, sempre existe
uma possibilidade de ela ter lucro, \(S_T>K\) e portanto seu preço deve
ser sempre maior que zero.

\begin{figure}
\centering
\includegraphics{img/trava_alta.png}
\caption{\label{fig:trava-alta}Perfil de lucro de uma trava de alta.}
\end{figure}

É claro que quanto mais ITM estejam as opções, maior seu preço e quanto
mais fora do dinheiro menor será seu valor até o limite inferior zero.
Se levarmos a diferença entre os \emph{strikes}, \(dK\) a zero temos
que:

\begin{equation}
\frac{\partial C}{\partial K}\leq 0
\end{equation}

Este é o limite de arbitragem para travas de alta ou, mais conhecido
pelo termo em inglês \emph{call spread no-arbitrage} e impõe que os
preços das calls devem ser uma função descrescente no \emph{strike}. De
forma equivalente e através da \protect\hyperlink{putcalparity}{paridade
compra-venda} este limite de arbitragem para as puts é:

\begin{equation}
\frac{\partial P}{\partial K}\geq 0
\end{equation}

\subsection{Arbitragem de borboleta}\label{arbitragem-de-borboleta}

Também deve ser imposta uma restrição na segunda derivada do preço das
opções em relação ao \emph{strike}, e esta é conhecida como limite de
arbitragem para borboletas. Vejamos porquê.

Considere uma estratégia do tipo borboleta, onde se compra uma quantia
de calls no \emph{strike} \(K-dK\), vende-se duas vezes esta quantia em
\(K\) e compra-se novamente um lote em \(K+dK\), o perfil de lucro desta
operação no vencimento está representado na figura \ref{fig:borboleta}.

\begin{figure}
\centering
\includegraphics{img/borboleta.png}
\caption{\label{fig:borboleta}Borboleta realizada com calls.}
\end{figure}

Seguindo a mesma linha de raciocínio anterior, como o \emph{payoff} da
borboleta é sempre não negativo também deve ser o seu valor para
qualquer período anterior a expiração. Se denotarmos \(\pi_B\) o valor
da borboleta, então \(\pi_B\geq0\).

Agora imagine que escalamos a estratégia de forma que um lote de compras
(na venda são dois lotes) seja de tamanho \(1/dK^2\), o valor para a
montagem desta operação deve ser, portanto:

\begin{equation}
\pi_B=\frac{C(K-dK)-2C(K)+C(K+dK)}{dK^2}
\end{equation}

E se levarmos ao limite em que \(dK\rightarrow 0\), a equação acima
torna-se justamente a segunda derivada do preço da call no \emph{strike}
\(K\).

\begin{equation}
\begin{aligned}
\frac{\partial^2 C(K)}{\partial K^2}=& \pi_B\\
\geq & 0
\end{aligned}
\label{eq:arbborboleta}
\end{equation}

Ou seja, os preços das calls são uma função \textbf{convexa} nos
\emph{strikes}. O mesmo raciocínio pode ser feito para uma borboleta com
puts e o resultado será equivalente, o preço das puts também deve ser
uma função convexa nos \emph{strikes}.

\subsection{Arbitragem de calendário}\label{arbitragem-de-calendario}

Passamos agora a analisar os limites de arbitragem na estrutura a termo
da superfície de volatilidade. A arbitragem de calendário normalmente é
expressa em termos de monotonicidade dos preços em relação ao período
para expiração. Ou seja, quanto maior o prazo de maturidade de uma opção
para um mesmo preço de exercício, maior deve ser seu valor.

É fácil de entender este limite com base nas probabilidades de
exercício. Como sabemos, em um
\protect\hyperlink{processos-estocasticos}{processo estocástico} do tipo
MBG a variância do processo cresce conforme a \textbf{raiz do tempo},
\(\sqrt{\tau}\). Quanto maior a variância do ativo subjacente, maior a
probabilidade deste alcançar um determinado preço, mais elevado ou não.
Assim, seja uma call ou put OTM quanto mais distante estiver seu prazo
de maturidade, maior a probabilidade de exercício e portanto, maior seu
valor.

Dado que a relação de \textbf{volatilidade total} implícita e preço de
uma opção também é direta e positiva, conforme demonstrado na Seção
\ref{smileprecificacao}, segue que a volatilidade total deve ser não
decrescente no tempo para expiração.

Esta relação pode ser expressa através da seguinte equação para uma call
precificada através de B\&S:

\begin{equation}
\frac{\partial C_{BS}(k, \theta(\tau))}{\partial \tau}=\partial_\theta C_{BS}\cdot\partial_\tau \theta \geq 0
\label{eq:arbcalendario}
\end{equation}

onde \(\partial_\theta C_{BS}\) é a derivada parcial do preço da call em
relação a volatilidade total implícita, que já demonstramos ser positiva
e \(\partial_\tau \theta\) é a derivada parcial da volatilidade total
implícita em relação ao tempo para maturidade que, portanto, deve ser
maior ou igual a zero para obedecer a restrição imposta ao preço da
call.

\section{Limites de inclinação}\label{limites-de-inclinacao}

Se mantivermos a volatilidade implícita constante para todos os
\emph{strikes}, os preços das calls no modelo B\&S devem ser
decrescentes. Por outro lado, para um \emph{strike} fixo, o preço de uma
call se eleva à medida que a volatilidade implícita aumenta. Suponha por
um momento que a volatilidade implícita varia com o \emph{strike} como é
o caso nos smiles. À medida que o \emph{strike} aumenta, se a
volatilidade implícita aumentar muito rapidamente, seu efeito sobre o
preço da call pode mais que compensar o declínio no preço devido a
elevação do preço de exercício e, assim, levar a um aumento líquido no
preço da opção. Isso violaria o requisito de que
\(\partial C /\partial K \leq 0\) e, portanto, leva a um limite superior
na taxa em que a volatilidade implícita pode aumentar com o strike.

Novamente, o mesmo raciocínio pode ser imposto para o lado das puts. A
volatilidade implícita não pode se elevar tão rapidamente quando os
\emph{strikes} se reduzem de forma que uma put de \emph{strike} menor
tenha valor mais elevado que outra que esteja mais próxima do dinheiro.

Finalmente, um sumário dos limites impostos a uma superfície de preços
de opções (calls no caso apresentado), que implicam em limites para a
superfície de volatilidade é apresentado abaixo\footnote{Retirado de
  \citet{Aurell2014}, p.~25.}:

\begin{enumerate}
\def\labelenumi{\arabic{enumi}.}
\tightlist
\item
  \(\partial_\tau C \geq 0\)
\item
  \(\lim\limits_{K\rightarrow\infty}C(K, \tau)=0\)
\item
  \(\lim\limits_{K\rightarrow-\infty}C(K, \tau)+K=a, \quad a \in \mathbb R\)
\item
  \(C(K, \tau)\) é convexa em \(K\)
\item
  \(C(K, \tau)\) é não-negativa
\end{enumerate}

\section{Distribuição implícita}\label{distribuicao-implicita}

O modelo B\&S é baseado na suposição que o ativo subjacente segue uma
distribuição log-normal em seus preços. Caso esta suposição fosse de
fato realizada no mercado, o smile de volatilidade seria uma reta
completamente horizontal, não haveria variação na volatilidade implícita
conforme o preço de exercício. Entretanto, esta não é a realidade dos
smiles e podemos fazer a pergunta inversa portanto, qual a distribuição
neutra ao risco que está implícita no smile de volatilidade?

Certamente não é uma log-normal. Na verdade, a densidade da distribuição
que está implícita em um smile nada mais é que a convexidade deste
smile, ou seja, sua segunda derivada em relação ao \emph{strike}. Esta
distribuição implícita também é por vezes chamada de RND \emph{(risk
neutral density)} e é muito útil para fazer a precificação de outras
opções que não são observadas no smile ou extrair probabilidades de
ocorrência de eventos precificadas pelo mercado.

Pode-se obter este resultado a partir da definição do valor de uma call
e é conhecido como a fórmula de \citet{Breeden1978}. O valor de uma call
é o valor esperado do \emph{payoff} terminal desta call ponderado pela
densidade neutra ao risco do subjacente. Ou seja:

\begin{equation}
C(S, t)=e^{-r\tau}\int\limits_{0}^\infty p(S,t,S_T,T)\max\{S_T-K, 0\}dS_T
\end{equation}

onde \(p(\cdot)\) é a densidade neutra ao risco e estamos supondo uma
taxa de juros livre de risco constante durante o período de vida da
opção. Como o \emph{payoff} da call é não linear, sendo zero para
qualquer valor de \(S_T \leq K\) e igual a \(S_T-K\) quando \(S_T > K\),
podemos escrever esta equação como:

\begin{equation}
C(S, t)=e^{-r\tau}\int\limits_{K}^\infty p(S,t,S_T,T)(S_T-K)dS_T
\end{equation}

que pode ser rearranjada, com alguma simplificação na notação, da
seguinte forma.

\begin{equation}
\begin{aligned}
\frac{\partial C}{\partial K}=& -e^{-r\tau}\int\limits_{K}^\infty p(S_T)dS_T\\
e^{r\tau}\frac{\partial C}{\partial K}=& \int\limits_{-\infty}^K p(S_T)dS_T\\
e^{r\tau}\frac{\partial^2 C}{\partial K^2}=& \ p(K)\\
\frac{\partial^2 C_B}{\partial K^2}=& \ p(K)
\end{aligned}
\label{eq:distimplicita}
\end{equation}

Onde usou-se a notação \(C_B\) para denotar a formulação de Black para o
preço de uma call. Ou seja, a segunda derivada em relação ao strike do
preço não descontado de uma call é a distribuição neutra ao risco do
ativo subjacente, e é válida para todos preços de exercício.

Portanto, se desejarmos saber qual a distribuição de probabilidades de
preços do ativo subjacente em uma data futura que possua vencimento de
opções, basta encontrarmos a convexidade do smile dos preços
\emph{forward} daquele vencimento\footnote{Simples em teoria, muito mais
  complicado na prática, com diversos problemas para a extrapolação do
  smile para \emph{strikes} extremos.}.

\section{Conclusão}\label{conclusao}

O modelo de Black-Scholes-Merton, pode ser considerado a pedra
fundamental para a precifição de opções. Entretanto, este modelo
apresenta uma séries de limitações que fazem com que os praticantes de
mercado utilizem outras técnicas neste mercado. Uma destas é o uso do
smile de volatilidade e sua interpretação como forma de precificar
opções e extrair informações implícitas nos preços.

A assimetria do smile e suas asas informam que as distribuições de
probabilidades para o ativo subjacente não são exatamente log-normais, e
podem apresentar discrepâncias significativas, especialmente nas caudas
da distribuição que muito interessam a gestão de risco, por exemplo.

Este foi um artigo denso, porém com vários conceitos importantes para a
compreensão do comportamento da superfície de volatilidade. A estrutura
a termo também é existente na volatilidade implícita e está limitada
pela ausência de arbitragem do tipo calendário. O smile de volatilidade,
que é uma fatia da superfície com prazo de expiração constante, possui
suas próprias limitações de forma, com a ausência de arbitragem do tipo
borboleta e limitações quanto a inclinação.

Por fim, foi demonstrado como a convexidade do smile de preços fornece a
distribuição implícita para os preços do ativo subjacente para a data de
expiração das opções.

\bibliography{library.bib}


\end{document}
